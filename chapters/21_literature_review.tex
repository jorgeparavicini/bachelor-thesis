\chapter{Relevant Academic Literature and Research}

Performance optimization is a crucial aspect of computer science, with numerous studies exploring its significance and potential areas of enhancement. This section provides a literature review on topics pivotal to this thesis - code generation and optimization, the impact of reflection on performance, the challenges in ASP.NET Core routing, and the current state of research regarding source generators. This exploration of the existing academic literature will identify gaps in recent research and pave the way for the original contributions of this bachelor's thesis.

\section{Code Generation and Optimization}

Code generation and optimization, being essential areas of research within computer science, have the power to augment system performance, reduce memory usage, and boost overall efficiency.

Early studies in this field primarily focused on compiler optimizations. Aho et al. \cite{Aho2007} offer a thorough study of various optimization techniques employed in modern compilers, such as constant folding, dead code elimination, and loop optimizations. Their work underscores the importance of these methods in enhancing the speed of compiled programs.

As the field has evolved, research has branched out to examine dynamic code generation and optimization. Kistler and Franz \cite{Kistler2003} put forward a system that generates code at load-time and continuously optimizes it at runtime. This approach allows the software to adjust to the exact hardware capabilities during execution, thereby overcoming modularization overheads through global optimization. Furthermore, it enables software to adapt to changing user behavior and incorporate dynamic link libraries at a later stage. While they caution that the cost-benefit ratio of continuous optimization can vary, it can lead to speed-ups of more than 120\% under ideal conditions. This indicates the potential of dynamic code generation and optimization when applied correctly.

Recently, machine learning has started to play a significant role in compiler optimization. Madhav, Singaravel, and Karmel \cite{Shreyas2021} discuss how compiler optimization techniques contribute to efficient computing by enabling developers to maximize hardware performance without significant cost implications. Of particular interest in their study is the prospect of machine learning revolutionizing compiler optimization, leading to a more sustainable and efficient computing experience for both developers and users.

The trajectory of code generation and optimization, from early compiler optimizations to recent advancements involving machine learning, demonstrates the ongoing efforts within the software engineering field to elevate software performance. Even as the techniques have evolved, the objective remains consistent - to create efficient, performant, and adaptable software. This historical context underscores the importance of our thesis, which tackles a modern instance of this enduring challenge. By examining the applications and implications of source generators in C\#, we contribute to this ongoing effort to shape and improve software performance. This endeavor remains as relevant today as it was in the early days of software engineering.

\section{Reflection Usage and Performance Implications}

Reflection is a mechanism that allows programs to inspect and manipulate their structure and behavior at runtime. While this capability can be highly beneficial, it also carries performance considerations.

Reflection operations necessitate the dynamic resolution of types, which requires the Common Language Runtime (CLR) in .NET (or equivalent in other languages) to load classes and information at runtime \cite{Tudose2013}. This contrasts with non-reflective code, which loads classes and info at compile time. As a result, reflective operations tend to have slower performance than non-reflective ones due to the additional overhead required for dynamic resolution. An extensive study by Tudose et al. \cite{Tudose2013} states that reflective operations are time-consuming, and developers should avoid frequently calling them if possible due to their performance overhead.

Despite these performance considerations, reflection remains a powerful technique widely used across various programming ecosystems due to its flexibility and adaptability. Specifically, in the .NET ecosystem, an investigation revealed that over 15\% of more than 6 million NuGets utilize reflective techniques \cite{Beaumont2022}. This widespread usage underscores reflection's inherent value and versatility in software development, even in the face of potential performance costs.

This dichotomy—reflection's widespread use and the associated performance penalties—highlights the necessity for alternative strategies. Pursuing mechanisms that provide reflection-like flexibility with more optimal performance characteristics is gaining importance.

The study \cite{Beaumont2022} emphasizes that understanding and employing reflection techniques are valuable for developers at all stages of their careers, yet also cautions about potential performance implications. This reality underscores the need for performant alternatives to support the significant portion of NuGets relying on reflection, potentially enhancing their performance while retaining their functionality.

\section{Performance Implications of ASP.NET Core Routing}

ASP.NET Core is a renowned technology for building web applications and APIs. Its appeal among developers is primarily influenced by its performance attributes - execution speed, delivery time, and startup time.

A comparison study by Kronis and Uhanova \cite{Kronis2018} delved into the performance characteristics of ASP.NET Core and Java EE. They created benchmarks as REST services that processed data in JSON format and ran these benchmarks on the Kestrel web server. Although their investigation did not directly address startup time, it provided valuable insights into the relative performance capabilities of these two platforms.

Karlsson \cite{Karlsson2021} took a slightly different approach, pitting ASP.NET Core and Express.js against each other, using both alongside a MySQL database to construct Web APIs. Karlsson's findings revealed that, in the context of a RESTful API, ASP.NET Core handled lower loads more efficiently, but Express.js had the upper hand when processing a larger volume of concurrent requests. Interestingly, with the new querying language GraphQL, Express.js either equaled or surpassed ASP.NET Core regarding response times and resource usage.

The action discovery process in ASP.NET Core, which utilizes reflection to associate actions with controller routes, might represent a potential performance bottleneck due to the well-known overheads linked with reflection \cite{Beaumont2022}. However, despite the importance of this process, it has been largely overlooked in research, leading to a lack of comprehensive understanding of its effects on application startup time and performance. Therefore, a more thorough investigation of this process is needed. Exploring ways to refine it, mainly to reduce its reliance on reflection, could lead to significant enhancements in startup times of ASP.NET Core applications, underlining the need for an empirical study.

\section{Source Generators and Performance Improvements}

In the relatively uncharted domain of source generators in C\#, Slimak's work \cite{Slimak2022} stands out. His thesis offers an in-depth guide to source generators. The thesis aims to help readers understand source generators, their purpose and use cases, and how they differentiate from other prevalent code generation options in C\#. The work further explores the use of source generators in enterprise applications and proposes a more complex framework that could generate standard boilerplate code, thus improving development efficiency. However, the thesis underscores source generators' current immaturity and challenges for integration into existing .NET tools, like Visual Studio and IntelliSense.

Supplementing the insights from Slimak's research, another study worth discussing is Franz's \cite{Franz2022}, entitled "Trends im Compilerbau." Though written in German, the paper provides a broader view of the trends shaping the field of compiler construction, with a particular focus on source generators and Just-In-Time Compilers. Using a dataset from the IEEE Xplore library, Franz emphasizes how active and vibrant the compiler construction research field remains, even with its considerable history. The emergence of new tools, like source generators, is sparking fresh inquiries. A key takeaway from Franz's study is the sustained focus on code optimizations. These efforts to enhance the output performance of compilers remain a primary concern in both practical and research contexts.

Adding to these scholarly resources, the book "Introducing .NET 6" by Nico Vermeir \cite{Vermeir2022} offers a practical perspective. Though not a research paper, the book contains a chapter titled ".NET Compiler Platform" that explains source generators, among other topics. Vermeir describes how the .NET Compiler Platform, previously known as Project Roslyn, provides strength and flexibility to .NET. Additionally, the author discusses source generators, highlighting how they leverage Roslyn to generate code at compile time and incorporate that generated code into the compilation process.

\section{Strengths, Weaknesses, and Gaps in the Existing Research}

The existing body of literature presents a wealth of information, with various studies investigating different aspects of code generation and optimization. This comprehensive coverage, spanning from early compiler optimizations \cite{Aho2007} to dynamic code generation techniques \cite{Kistler2003}, and even to the incorporation of machine learning \cite{Shreyas2021}, offers a solid foundation for understanding the significance of software optimization. Additionally, the studies delve into the performance implications of reflection and its extensive usage despite the associated overhead \cite{Tudose2013, Beaumont2022}.

Simultaneously, focused examinations such as Slimak's thesis \cite{Slimak2022} and Franz's paper \cite{Franz2022} shine a light on practical applications of source generators and ongoing trends in compiler construction. The literature also includes a comparative performance analysis of ASP.NET Core against other frameworks \cite{Kronis2018, Karlsson2021}, further enriching our understanding of this field.

However, specific gaps and limitations persist in the current research landscape. While the intersection of machine learning and compiler optimization is broached \cite{Shreyas2021}, there's room for a deeper exploration. Moreover, despite acknowledging the reflection's performance cost, alternatives that retain its flexibility without compromising performance remain underexplored.

Significantly, the startup phase of ASP.NET Core applications, potentially impacted by action discovery and routing mechanism performance, has received limited attention. While comparative studies shed light on ASP.NET Core's request handling capabilities, more granular studies focused on its startup phase, particularly the action discovery process, are noticeably absent.

There's also a noted inconsistency regarding the efficacy of continuous optimization for dynamic code generation \cite{Kistler2003}, suggesting a need for more research to discern the specific conditions that maximize these techniques' benefits. Lastly, the immaturity of source generators and their integration challenges with existing .NET tools, as highlighted by Slimak \cite{Slimak2022}, underscores the need for more dedicated research.

\section{Contribution of this thesis}

This thesis contributes to the current body of knowledge in several ways, bridging existing gaps and enhancing our understanding of critical issues about code generation, optimization, and performance improvement in software development.

Firstly, this research expands on the application of source generators in the .NET ecosystem, primarily focusing on performance optimization. Despite the promising potential of source generators, previous research indicates that this technology is currently in a state of relative immaturity \cite{Slimak2022}. This thesis contributes a systematic analysis of source generators, evaluating their strengths, weaknesses, and application areas in the context of software performance improvement.

Secondly, this thesis critically investigates the impact of ASP.NET Core's routing mechanisms on performance—a subject that has received limited attention in the existing literature. By thoroughly examining routing mechanisms and their relationship to startup time, the study uncovers new ways to enhance the performance of ASP.NET Core applications.

Lastly, concerning reflection, existing research has highlighted its associated performance overhead. This study delves into a practical approach to alleviate these performance implications without forsaking the benefits of reflection. Rather than seeking alternatives to reflection, the study attempts to reduce the need for reflection at runtime, thereby preserving its functionality and versatility.