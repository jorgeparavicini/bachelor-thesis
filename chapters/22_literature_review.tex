\chapter{Preliminary Literature Review}

Using C\# source generators has received little attention in the existing research. While a few books \cite{Franz2022TrendsCompilerbau, Microsoft2022SourceGenerators, Vermeir2022.NETPlatform} are available on the implementation of C\# source generators and their integration in Roslyn; the official documentation \cite{Microsoft2022SourceGenerators} is lacking in detail. There are some papers on C\# source generators. However, they focus on the potential use cases and a general understanding of the feature \cite{Slimak2022SourceSLIMAK}. Conversely, this bachelor thesis will focus on the possible performance gain when using C\# source generators for reflected code.

Research on the performance penalty of reflective code in general and other programming languages is widely available. There have been various studies \cite{Forman2005EvaluatingPerformance, Halloway2001ReflectionInformIT, JavaReflection2013} which measure the performance of reflection in Java applications. In some of these studies, it has been explicitly stated that the development speed gained from using reflection outweighs the performance hit \cite{Halloway2001ReflectionInformIT}. However, with the rise of globalization and the need to process millions of user requests as fast as possible, every possible performance gain becomes crucial. Regardless of the development speed, all studies concluded that reflection imposes a significant performance penalty \cite{Forman2005EvaluatingPerformance, Halloway2001ReflectionInformIT, JavaReflection2013}. Therefore, it is important to investigate alternative approaches that can mitigate this penalty.

This bachelor thesis aims to contribute to the field of software engineering and performance optimization by investigating the potential benefits of using C\# source generators for reflected code. The existing research on reflection \cite{Forman2005EvaluatingPerformance, Halloway2001ReflectionInformIT, JavaReflection2013} mostly focuses on Java, which makes it difficult to determine the potential gains for C\#. Therefore, we aim to fill this gap in the literature by conducting a controlled experiment and comparing the observed runtime performance to the existing research. The results will provide a deeper understanding of the trade-off between build time and runtime performance in C\# applications. By doing so, we hope to provide useful insights for developers who strive to balance performance and development time in their applications.