\chapter{Conclusion and Outlook}

The primary objective of this thesis was to investigate the impact of using C\# source generators to convert reflective routing operations into static code in ASP.NET Core, focusing on its effect on runtime performance, memory usage, and build duration.

Our key findings indicate that static action discovery can reduce startup times and memory usage, enhancing the overall performance of ASP.NET Core applications. However, this improvement is offset by increased build times, forming a hurdle for the widespread adoption of this technique.

Despite promising results, we acknowledge several limitations in our study. The tests were conducted in a minimal setup, which may not entirely replicate complex, real-world scenarios. Potential discrepancies between static and dynamic approaches and unaccounted variables affecting startup time were identified. Nevertheless, the study has shown that C\# source generators have the potential to enhance ASP.NET Core application performance.

The results of this study pave the way for several avenues of exploration in future research. Our work was focused on static action discovery in web APIs, specifically those that do not rely heavily on properties. A progression would be to extend this investigation to include web applications, such as Blazor or Razor, which require properties for their bindings. Understanding how our static action discovery approach fares in these scenarios would be insightful.

Additionally, our study implemented the action discovery using incremental source generators, the successor to the traditional, now deprecated, source generators. It would be intriguing to assess the impact of the action discovery if it had been implemented using these older source generators.

Moreover, our test environment was carefully controlled, focusing on minimal setups to isolate the changes. Another line of inquiry would be to investigate how the static action discovery holds up in a more complex application environment with a more complicated startup phase.

On the technical front, an area of potential exploration is the challenge of deterministically finding out the values of constructor arguments or properties at compile time. This is often a difficult task, and improvements in this area could potentially eliminate the need for evaluating these properties and arguments at runtime, which we're trying to avoid. While our approach already eliminates the need for reflection, this is an area where we could push the performance benefits even further.

Finally, our research has also opened the doors for a broader exploration into the potential of source generators. We've found that source generators have immense potential for reducing code complexity, particularly in codes that use reflection. A study by Beaumont and Bastaki \cite{Beaumont2022} revealed that around 15\% of NuGet packages use reflection. Considering this, source generators could reduce the need for reflection and potentially enhance the performance of the entire .NET Ecosystem.
