\chapter{Introduction}

New tools and technologies emerge in the constantly evolving software development landscape, promising to streamline development processes and bolster efficiency. One such recent innovation is C\# source generators, a component of the .NET compiler platform that allows developers to generate source code dynamically at compile time. Source generators can inspect and manipulate the code's syntax tree, facilitating the generation of additional source code compiled alongside the existing one.

Introduced in 2020 with the release of C\# 9.0, source generators were a much-awaited tool, a fresh alternative to traditional code generation techniques, which often proved inflexible and maintenance-intensive. With the advent of source generators, developers now have a new instrument in their arsenal, promising to enhance code quality and simplify the development process.

This bachelor's thesis explores the potential benefits and limitations of utilizing C\# source generators in application development. As a relatively new concept, there is much to be discovered about the effectiveness of source generators and the best practices for their efficient use. The thesis aims to fill this knowledge gap by empirically evaluating the application of C\# source generators for boosting runtime performance.

The motivation for this work stems from two main sources: the increasing demand for high-performing enterprise applications capable of complex data processing and an eagerness to explore and understand new .NET technologies, specifically source generators. This thesis hopes to provide valuable insights and practical knowledge for developers aiming to optimize their .NET applications by examining how source generators can be harnessed to improve application performance.

The primary objective of this research is to examine whether employing C\# source generators can improve the runtime performance of ASP.NET web applications. In the fast-paced business world, balancing performance and development speed is crucial. For instance, reflection, a modern approach that allows a program to introspect its structure and behavior at runtime, significantly enhances development speed. However, this comes at the cost of runtime performance. C\# source generators offer a solution to this issue by generating static that would otherwise use reflection during compilation, thus removing the need for runtime reflection.

The guiding research question for this investigation is:

\begin{enumerate}[label=\textbf{RQ.\arabic*}:, leftmargin=*, labelindent=1em]
    \item "How does using C\# source generators to convert reflective action discovery into static code in ASP.NET Core impact startup performance, memory usage, and build time duration?"
\end{enumerate}

The scope of this research is limited to using C\# source generators for generating static code from reflective code, focusing on effects on runtime, build time performance, and memory usage. The research will not consider other factors like code complexity or other potential source generator uses. Additionally, this study will be limited to C\# applications without considering applications written in different programming languages.
