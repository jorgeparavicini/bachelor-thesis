\chapter{Introduction}

In the rapidly evolving software development landscape, new technologies continuously reshape our working methods, each promising to streamline processes and augment efficiency. One promising technology is C\# source generators, introduced as part of the .NET compiler platform. These tools allow for the dynamic generation of source code during compilation, opening possibilities for developers worldwide.

The introduction of C\# source generators with the release of C\# 9.0 in 2020 marked a shift from traditional code generation techniques that were often inflexible and challenging to maintain. These new tools offer developers a pathway to enhanced code quality, simplified development processes, and improved application performance.

This bachelor's thesis explores these new horizons by examining the potential benefits and limitations of utilizing C\# source generators in web application development. As an emergent concept, there is much to be learned about the effectiveness of source generators and the optimal strategies for their use. By empirically evaluating their impact on runtime performance, this thesis strives to bridge this knowledge gap.

The increasing demand for high-performing enterprise applications capable of handling complex data processing and the eagerness to unravel the intricacies of new .NET technologies motivates this study. This research aims to equip developers with valuable insights into optimizing their .NET applications by investigating how source generators can be effectively harnessed to improve performance.

Central to this research is whether employing C\# source generators can enhance the startup performance of ASP.NET web applications. This becomes particularly significant in a world where performance and development speed are often at odds. Reflection, a modern technique allowing programs to inspect their structure and behavior at runtime, increases development speed but often at the expense of runtime performance.

C\# source generators offer a solution to this issue by generating static code that would otherwise use reflection during compilation, thus removing the need for runtime reflection. This research focuses explicitly on innovating a new method for action discovery, an essential part of processing web requests. By utilizing source generators to perform this task statically, it is anticipated that the startup performance of ASP.NET applications will improve. 

The guiding research question for this bachelor's thesis is:

\begin{enumerate}[label=\textbf{RQ.\arabic*}:, leftmargin=*, labelindent=1em]
\item "How does using C\# source generators to convert reflective action discovery into static code in ASP.NET Core impact startup performance, memory usage, and build time duration?"
\end{enumerate}

While we explore the potential of C\# source generators, it's also vital to be aware of their limitations. They can't modify existing code; they only append new code during compilation. Also, poorly designed source generators could extend compile times, as their runtime forms part of the overall compile time. They work on a syntax tree representation of the code and lack access to the results of prior compiler phases. Moreover, their use demands a paradigm shift that could potentially increase code complexity, posing a learning curve for developers.

The scope of this research is limited to using C\# source generators for generating static code from reflective code, focusing on effects on runtime, build time performance and memory usage. The research will not consider factors like code complexity or other potential source generator uses. Additionally, this study will be limited to C\# applications without considering applications written in different programming languages.