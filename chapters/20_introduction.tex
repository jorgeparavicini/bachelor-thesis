\chapter{Introduction}

In the dynamic world of software development, new tools and technologies continually surface, each offering ways to make the development process more efficient and streamlined. A recent and promising innovation in this sphere is C\# source generators. Part of the .NET compiler platform, these source generators permit the dynamic generation of source code during the compilation. The ability to generate additional source code and compile it in tandem with the existing code opens up new possibilities for developers.

C\# source generators were introduced with the launch of C\# 9.0 in 2020. This development was a welcome change from traditional code generation techniques that were less flexible and difficult to maintain. The arrival of source generators provided developers with a new tool that had the potential to enhance code quality, simplify the development process, and improve application performance.

This bachelor's thesis explores the potential benefits and limitations of utilizing C\# source generators in web application development. As a relatively new concept, there is much to be discovered about the effectiveness of source generators and the best practices for their efficient use. The thesis aims to fill this knowledge gap by empirically evaluating the application of C\# source generators for boosting runtime performance.

The motivation for this work stems from two primary sources: the increasing demand for high-performing enterprise applications capable of complex data processing and an eagerness to explore and understand new .NET technologies, specifically source generators. This thesis hopes to provide valuable insights and practical knowledge for developers aiming to optimize their .NET applications by examining how source generators can be harnessed to improve application performance.

The primary objective of this research is to examine whether employing C\# source generators can improve the startup performance of ASP.NET web applications. In the fast-paced business world, balancing performance and development speed is crucial. For instance, reflection, a modern approach that allows a program to introspect its structure and behavior at runtime, significantly enhances development speed. However, this comes at the cost of runtime performance. 

C\# source generators offer a solution to this issue by generating static code that would otherwise use reflection during compilation, thus removing the need for runtime reflection. This research focuses explicitly on innovating a new method for action discovery, an essential part of processing web requests. By utilizing source generators to perform this task statically, it is anticipated that the startup performance of ASP.NET applications will improve. 

The guiding research question for this bachelor's thesis is:

\begin{enumerate}[label=\textbf{RQ.\arabic*}:, leftmargin=*, labelindent=1em]
    \item "How does using C\# source generators to convert reflective action discovery into static code in ASP.NET Core impact startup performance, memory usage, and build time duration?"
\end{enumerate}

While we explore the potential of C\# source generators, it's also vital to be aware of their limitations. They can't modify existing code; they only append new code during compilation. Also, poorly designed source generators could extend compile times, as their runtime forms part of the overall compile time. They work on a syntax tree representation of the code and lack access to the results of prior compiler phases. Moreover, their use demands a paradigm shift that could potentially increase code complexity, posing a learning curve for developers.

The scope of this research is limited to using C\# source generators for generating static code from reflective code, focusing on effects on runtime, build time performance, and memory usage. The research will not consider factors like code complexity or other potential source generator uses. Additionally, this study will be limited to C\# applications without considering applications written in different programming languages.
