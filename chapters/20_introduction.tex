\chapter{Introduction}

In the ever-changing software development landscape, new technologies and tools are continuously emerging to make the development process more efficient and streamlined. One of these recent innovations is C\# source generators. These source generators are part of the .NET compiler platform and enable developers to dynamically generate source code at compile time. Additionally, they allow developers to write code that inspects and manipulates the syntax tree of their code, which in turn enables the generation of additional source code to be compiled alongside the existing code.

C\# source generators were first introduced with the release of C\# 9.0 in 2020. Before their introduction, developers had to rely on traditional code generation techniques like T4 templates or code-behind files, which were less flexible and more challenging to maintain compared to source generators. With C\# 9.0, developers gained access to a new tool that can help improve code quality and streamline development processes.

This bachelor thesis delves into the potential advantages and limitations of using C\# source generators in application development. Since source generators are a relatively new and unexplored concept, there is a gap in the literature concerning their effectiveness and best practices for efficient use. This bachelor thesis aims to fill this gap by empirically evaluating using C\# source generators to enhance runtime performance.

\section{Motivation}

This work is motivated by two primary factors. First, this research drives the increasing demand for fast and efficient enterprise applications. As applications must handle complex data processing, runtime performance has become crucial in ensuring their success. Achieving improved performance without sacrificing development efficiency is a challenging yet essential goal for software developers.

Second, a strong motivator is a keen interest in exploring and understanding new technologies in the .NET environment, particularly source generators, and how they can be harnessed to improve application performance. C\# source generators present a promising solution for balancing the need for high-performance applications while maintaining a manageable development process. This thesis aims to contribute valuable insights and practical knowledge for developers looking to optimize their .NET applications by investigating the potential benefits and limitations of using C\# source generators in application development.

\section{Research Objectives}

The primary objective of this bachelor thesis is to explore whether using C\# source generators can enhance the runtime performance of ASP.NET web applications. In today's fast-paced business world, enterprises aim to deliver quick and efficient information to their users while also saving time on development. However, achieving higher application speeds often requires investing more time in development and using more complex technologies. For instance, game development often necessitates low-level languages like C++ or Rust to meet performance demands. In contrast, businesses providing data through web applications prioritize time-to-market, opting for slower languages like C\# or Java that offer faster development speeds than C++.

Striking a balance between performance and development time is crucial. Reflection is a modern approach to enhancing application development speed. It enables a program to inspect and manipulate its structure and behavior at runtime, such as class and method definitions. For instance, reflection is widely used in modern web applications during the bootstrap process to identify endpoints. However, this approach has drawbacks, as it must be performed every time the application starts, and reflection is inherently time-consuming. C\# source generators offer a solution to this issue by converting code with reflection into static code during compilation, eliminating the need for reflection at runtime.

\section{Research Question}

This research aims to comprehensively understand the potential of C\# source generators as a tool for improving web application performance. By exploring the trade-offs between using reflection at runtime and generating static code at compile time, this research provides insights into the future of software development and how developers can leverage new features in programming languages to build better and faster applications. The research questions guiding this investigation are:

\begin{enumerate}[label=\textbf{RQ.\arabic*}:, leftmargin=*, labelindent=1em]
    \item "How does using C\# source generators to convert reflective routing operations into static code in ASP.NET Core impact runtime performance, memory usage, and build time duration?"
\end{enumerate}

\section{Scope and Limitations}

The scope of this research is limited to using C\# source generators to generate static code for reflected code. The research will focus on applications' runtime, build-time performance, and memory usage but will not consider other factors, such as code complexity. The research will not cover other approaches to improving runtime performance, such as optimization techniques or alternative programming languages. Furthermore, the scope will be limited to exploring the potential of C\# source generators to enhance runtime performance and will not delve into other potential uses for this feature. Lastly, the study will be limited to applications written in C\# and not consider applications written in other programming languages.
