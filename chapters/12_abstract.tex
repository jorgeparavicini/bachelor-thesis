\begin{zusammenfassung}
Bei der Entwicklung von Webanwendungen ist die Optimierung der Startzeiten ein Schlüsselaspekt für effizientes Ressourcenmanagement und Kostenreduktion. Diese Arbeit untersucht einen Ansatz zur Verbesserung der Startgeschwindigkeit in mit ASP.NET Core erstellten Webanwendungen. Im Fokus der Studie steht der Prozess der 'Action Discovery', also der Weg des Systems, verschiedene Routen innerhalb einer Webanwendung zu identifizieren. Traditionell findet dieser Prozess während des Starts der Anwendung statt. In einem Experiment, das diese traditionelle Methode mit einem neuen Ansatz vergleicht, wurde ein Tool namens C\# Source Generator verwendet, um den Prozess der Action Discovery in die Bauphase der Anwendung zu verlagern. Die Auswirkungen dieser Verschiebung auf die Startgeschwindigkeit, den Speicherverbrauch und die Dauer des Bauprozesses wurden gründlich untersucht. Die Ergebnisse zeigen erhebliche Verbesserungen bei der Startgeschwindigkeit, insbesondere bei grösseren Anwendungen, und eine Abnahme des Speicherverbrauchs beim Start, wenn auch mit dem Kompromiss von längeren Bauzeiten. Diese Studie bietet daher einen experimentellen Einblick in die potenziellen Vorteile und Kompromisse dieses innovativen Ansatzes und bereitet den Weg für zukünftige Forschungen zur Optimierung dieser Ergebnisse.
\end{zusammenfassung}

\begin{abstract}
In web application development, optimizing startup times is a key aspect of efficient resource management and cost reduction. This thesis investigates an approach to improve startup speed in web applications created with ASP.NET Core. The study focuses on the 'action discovery' process, the system's way of identifying different routes within a web application. Traditionally, this process occurs during application startup. In an experiment comparing this traditional method with a new approach, a tool called a C\# source generator was used to shift the action discovery process to the application's building phase. The effects of this shift on startup speed, memory usage, and build duration were thoroughly examined. Results indicate significant improvements in startup speed, especially for larger applications, and a decrease in memory usage during startup, albeit with the trade-off of longer build times. This study thus provides an experimental insight into this innovative approach's potential benefits and trade-offs, laying the groundwork for future research to optimize these outcomes.
\end{abstract}