\begin{zusammenfassung}
In dieser Arbeit wird der Einfluss der Verwendung einer statischen Methode zur Identifikation von Aktionen auf die Leistung von ASP.NET Core-Anwendungen untersucht. Im Fokus steht dabei die Anwendung von C\# Source Generators, um den üblicherweise auf Reflexion basierenden Prozess in einen statischen Prozess umzuwandeln. Die Studie analysiert, inwiefern diese Änderung die Startzeit der Anwendung, den Speicherverbrauch und die Build-Dauer beeinflusst. Die Ergebnisse zeigen, dass der statische Ansatz die Startzeit, vor allem bei grösseren Anwendungen, erheblich beschleunigt. Kleinere Anwendungen profitieren ebenfalls, auch wenn die absolut eingesparte Zeit gering ist. Der Speicherverbrauch während des Starts reduziert sich bei der statischen Methode, allerdings kommt es zu einer Verlängerung der Build-Zeit, insbesondere bei grösseren Anwendungen. Die Resultate unterstreichen die potenziellen Vorteile des statischen Ansatzes in ASP.NET Core-Anwendungen, weisen aber auf einen Kompromiss zwischen einer schnelleren Startzeit und einer längeren Build-Zeit hin. Die Arbeit empfiehlt zukünftige Forschungen, die sich auf die Optimierung dieses Kompromisses konzentrieren und die Auswirkungen auf unterschiedliche Anwendungstypen weiter erforschen.
\end{zusammenfassung}

\begin{abstract}
This thesis explores the impact of static action discovery on the performance of ASP.NET Core applications. It evaluates the use of C\# source generators to convert the action discovery process from a reflection-based mechanism to a static one. The study investigates how this conversion affects the application's startup time, memory use, and build time. The findings show that static action discovery significantly speeds up the startup time, especially in larger applications. Smaller applications also benefit, although the absolute time saved is small. Memory usage during startup is lower with static action discovery, but there's an increase in build times, which is particularly evident for larger applications. The results highlight the potential benefits of static action discovery in ASP.NET Core applications while revealing a trade-off between faster startup and longer build times. The thesis suggests that further research could focus on optimizing this trade-off and exploring the impact on different types of applications.
\end{abstract}