\chapter{Conclusion}

In conclusion, the main objective of this bachelor thesis is to determine whether the use of C\# source generators to generate static code for reflected code significantly improves runtime performance. A systematic literature review is conducted to answer this question, followed by a controlled experiment where applications are developed and benchmarked. The expected outcome of this research is that the runtime performance will be improved at the cost of build time duration.

This thesis is structured as follows: In the first section, a comprehensive literature review is conducted to provide an overview of the current state of research in the source generators and reflection field. The second section details the controlled experiment conducted, including the development of applications and the benchmarking process. In the final section, the results are analyzed, and the research questions are answered.

This thesis's significance lies in its contribution to the field of software engineering and performance optimization. The research aims to better understand the trade-off between build time and runtime performance. Furthermore, the experiments and analysis results are expected to offer valuable insights into using C\# source generators for performance optimization.

\section{Future Work}
\begin{itemize}
    \item Suggest areas for further research or improvements, such as exploring other use cases for C\# source generators or extending the approach to other frameworks and languages.
\end{itemize}